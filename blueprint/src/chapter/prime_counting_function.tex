\chapter{The Prime Counting Function}
\label{chap:prime_counting_function}

\unintegrated

Let $\Lambda(n)$ denote the von Mangoldt function, i.e. $\Lambda(n) = \log p$ if $n = p^m$ where p is prime and $m$ is a positive integer, and $\Lambda(n) = 0$ otherwise. 

\begin{definition}
For all $x \ge 1$ define the prime counting functions $\psi(x)$, $\theta(x)$ and $\pi(x)$ as
\[
\psi(x) := \sum_{n \le x}\Lambda(n),\qquad \theta(x) := \sum_{p \le x}\log p,\qquad \pi(x) := \sum_{p \le x}1
\]
where the first sum is over positive integers $n$ and the last two sums are over primes $p$.
\end{definition}

These functions, particularly $\pi(x)$, are central to number theory because they measure the long range distribution of prime numbers among the integers. A well-known result is the prime number theorem.

\begin{theorem}[Prime number theorem]
As $x \to \infty$, 
\[
\pi(x) \sim \frac{x}{\log x} \sim \operatorname{li}(x) := \int_2^{\infty}\frac{dt}{\log t}.
\]
\end{theorem}

The following are equivalent formulations of the prime number theorem.

\begin{theorem}
As $x \to \infty$, one has $\psi(x) \sim x$ and $\theta(x) \sim x$.
\end{theorem}

\section{Error Bounds for prime counting functions}
In addition to their asymptotic behaviour, various bounds on the deviation from their respective asymptotics are known. The current best-known error bounds are derived from zero-free regions of the Riemann zeta function $\zeta(s)$. 

\begin{theorem}[Korobov--Vinogradov estimate]
There exists a positive constant $A$, such that
\begin{align*}
\psi(x) - x, \; \theta(x) - x, \; \pi(x) - \operatorname{li}(x) \ll x\exp\left(-A\frac{(\log x)^{3/5}}{(\log\log x)^{1/5}}\right).
\end{align*}
\end{theorem}

\Cref{prime-error-table} lists the historical progression on estimates of $\pi(x)$.

\begin{table}[ht]
    \caption{Historical estimates of $\pi(x)$, for $x$ sufficiently large.}
    \centering
    \renewcommand{\arraystretch}{2.2}
    \begin{tabular}{|c|c|}
    \hline
    Reference & Estimate of $\pi(x)$\\
    \hline
    Chebyshev & $c_1 \dfrac{x}{\log x} \leq \pi(x) \leq c_2 \dfrac{x}{\log x}$ for some constants $0 < c_1 < 1 < c_2$\\
    \hline
    de la Vallee Poussin, Hadamard & $\pi(x) = \operatorname{li}(x)(1 + o(1))$\\
    \hline
    de la Vallee Poussin & $\pi(x) - \operatorname{li}(x) \ll x\exp(-A\sqrt{\log x})$ for some $A > 0$\\
    \hline
    Littlewood & $\pi(x) - \operatorname{li}(x) \ll x\exp(-A\sqrt{\log x\log\log x})$ for some $A > 0$\\
    \hline 
    Korobov, Vinogradov & $\pi(x) - \operatorname{li}(x) \ll x\exp(-A(\log x)^{3/5}(\log\log x)^{-1/5})$ for some $A > 0$\\
    \hline
    \end{tabular}\label{prime-error-table}
\end{table}

\Cref{prime-error-table-explicit} lists the historical progression of explicit bounds on $\pi(x)$.

\begin{table}[ht]
    \caption{Historical explicit estimates of $\pi(x)$.}
    \centering
    \renewcommand{\arraystretch}{2.2}
    \begin{tabular}{|c|c|}
    \hline
    Reference & Estimate of $\pi(x)$\\
    \hline
    Classical & $\pi(x) > \dfrac{x}{\log x + 2}$ for sufficiently large $x$\\
    \hline
    Dusart (2010) & $\dfrac{x}{\log x} \left(1 + \dfrac{1}{\log x}\right) < \pi(x) < \dfrac{x}{\log x} \left(1 + \dfrac{1.25506}{\log x}\right)$ for suff. large $x$\\
    \hline
    \end{tabular}\label{prime-error-table-explicit}
\end{table}

\section{Alternative Approximations}
Besides $x/\log x$, more accurate approximations for $\pi(x)$ exist, such as Riemann’s R-function:
\[
R(x) = \sum_{n=1}^{\infty} \frac{\mu(n)}{n} \operatorname{Li}(x^{1/n}),
\]
where $\mu(n)$ is the Möbius function. This provides an integral-based approach to estimating $\pi(x)$ with better precision.

\section{Computational Aspects}
Computing $\pi(x)$ efficiently is a key problem in number theory. Several algorithms exist, including:
\begin{itemize}
    \item **Direct sieving methods** (e.g., the Sieve of Eratosthenes)
    \item **Meissel-Lehmer algorithm**, which divides the computation into smaller, manageable parts
    \item **Lagarias-Miller-Odlyzko method**, which uses integral approximations
\end{itemize}
Modern implementations allow computation of $\pi(x)$ for very large $x$, with values exceeding $10^{18}$ computed in practice.

\section{Future Research Directions}
Some active areas of research related to $\pi(x)$ include:
\begin{itemize}
    \item **Improving error bounds** using explicit estimates on zeta zeros.
    \item **Connections to the Riemann Hypothesis**, which predicts even sharper estimates for $\pi(x)$.
    \item **Efficient computation techniques** for evaluating $\pi(x)$ at extremely large values.
\end{itemize}
Further progress in these areas could lead to a deeper understanding of prime number distribution.

\section{References}
For further reading and more detailed proofs, consult:
\begin{itemize}
    \item Dusart, P. (2010). Estimates of some functions over primes without Riemann Hypothesis.
    \item Wikipedia page on the Prime Counting Function: \url{https://en.wikipedia.org/wiki/Prime-counting_function}
    \item TME-EMT Article on Prime Counting Function: \url{https://tmeemt.github.io/Chest/Articles/Art01.html}
\end{itemize}
