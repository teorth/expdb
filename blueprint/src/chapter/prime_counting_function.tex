\chapter{The Prime Counting Function}
\label{chap:prime_counting_function}

\unintegrated

Let $\Lambda(n)$ denote the von Mangoldt function, i.e. $\Lambda(n) = \log p$ if $n = p^m$ where p is prime and $m$ is a positive integer, and $\Lambda(n) = 0$ otherwise. 

\begin{definition}
For all $x \ge 1$ define the Chebyshev prime counting functions $\psi(x)$, $\theta(x)$ and $\pi(x)$ as
\[
\psi(x) := \sum_{n \le x}\Lambda(n),\qquad \theta(x) := \sum_{p \le x}\log p,\qquad \pi(x) := \sum_{p \le x}1
\]
where the first sum is over positive integers $n$ and the last two sums are over primes $p$.
\end{definition}

These functions, particularly $\pi(x)$, are central to number theory because they measure the long range distribution of prime numbers among the integers. A well-known result is the prime number theorem.

\begin{theorem}[Prime number theorem]
As $x \to \infty$, 
\[
\pi(x) \sim \frac{x}{\log x} \sim \operatorname{li}(x) := \int_2^{\infty}\frac{dt}{\log t}.
\]
\end{theorem}

The following are equivalent formulations of the prime number theorem.

\begin{theorem}[Prime number theorem, alternative formulations]
As $x \to \infty$, one has $\psi(x) \sim x$ and $\theta(x) \sim x$.
\end{theorem}

\section{Error Bounds for prime counting functions}
In addition to their asymptotic behaviour, various bounds on the deviation from their respective asymptotics are known. The current best-known error bounds are derived from zero-free regions of the Riemann zeta function $\zeta(s)$. The relation between zeroes of $\zeta(s)$ and error bounds for prime counting functions are illustrated through von Mangoldt's explicit formula: for all non-integer $x > 0$, one has
\[
\psi(x) = x - \sum_{\rho}\frac{x^\rho}{\rho} - \log 2\pi - \frac{1}{2}\log(1 - x^{-2}),
\]
where $\rho$ runs through all non-trivial zeroes of $\zeta(s)$.

\begin{theorem}[Korobov--Vinogradov estimate]
There exists a positive constant $A$, such that
\begin{align*}
\psi(x) - x, \; \theta(x) - x, \; \pi(x) - \operatorname{li}(x) \ll x\exp\left(-A\frac{(\log x)^{3/5}}{(\log\log x)^{1/5}}\right).
\end{align*}
\end{theorem}

\Cref{prime-error-table} lists the historical progression on estimates of $\pi(x)$.

\begin{table}[ht]
    \caption{Historical estimates of $\pi(x)$, for $x$ sufficiently large.}
    \centering
    \renewcommand{\arraystretch}{2.2}
    \begin{tabular}{|c|c|}
    \hline
    Reference & Estimate of $\pi(x)$\\
    \hline
    Chebyshev & $c_1 \dfrac{x}{\log x} \leq \pi(x) \leq c_2 \dfrac{x}{\log x}$ for some constants $0 < c_1 < 1 < c_2$, i.e. $\pi(x) = \dfrac{x}{\log x}(1 + O(1))$\\
    \hline
    de la Vall\'{e}e Poussin \cite{de_la_vallee_poussin_recherches_1896}, Hadamard \cite{hadamard_distribution_1896} & $\pi(x) = \dfrac{x}{\log x}(1 + o(1))$\\
    \hline
    de la Vall\'{e}e Poussin \cite{de_la_vallee_poussin_fonction_1899} & $\pi(x) = \operatorname{li}(x) + O(x\exp(-A\sqrt{\log x}))$ for some $A > 0$\\
    \hline
    Littlewood \cite{littlewood_researches_1922} & $\pi(x) = \operatorname{li}(x) + O(x\exp(-A\sqrt{\log x\log\log x}))$ for some $A > 0$\\
    \hline 
    Korobov, Vinogradov \cite{vinogradov_eine_1958} & $\pi(x) = \operatorname{li}(x) + O(x\exp(-A(\log x)^{3/5}(\log\log x)^{-1/5}))$ for some $A > 0$\\
    \hline
    \end{tabular}\label{prime-error-table}
\end{table}

Under the Riemann hypothesis, stronger error bounds are known. 
\begin{theorem}[\cite{koch_sur_1901}]
If the Riemann hypothesis is true, then
\[
\psi(x) - x,\; \theta(x) - x \ll x^{1/2}(\log x)^2,\qquad \pi(x) - \operatorname{li}(x) \ll x^{1/2}\log x.
\]
\end{theorem}

\section{Omega results}

In the opposite direction, it is known that 
\begin{theorem}[Schmidt \cite{schmidt_uber_1903}]
As $x \to \infty$,
\[
\psi(x) = x + \Omega(x^{1/2}).
\]
\end{theorem}
This can be improved slightly conditioned on the Riemann hypothesis.
\begin{theorem}[Littlewood \cite{littlewood_sur_1914}]
If the Riemann hypothesis is true, then as $x \to \infty$,
\[
|\pi(x) - \operatorname{li}(x)| = \Omega\left(x^{1/2}\frac{\log\log\log x}{\log x}\right).
\]
\end{theorem}

Furthermore it is also known that
\begin{theorem}[Grosswald \cite{grosswald_sur_1965}]
If 
\[
\theta = \sup_{\rho: \zeta(\rho) = 0}\Re \rho > 1/2
\]
then as $x \to \infty$,
\[
\psi(x) = x + \Omega(x^{\theta}).
\]
\end{theorem}

\section{Computational Aspects}
Computing $\pi(x)$ efficiently is a key problem in number theory. Several algorithms exist, including:
\begin{itemize}
    \item **Direct sieving methods** (e.g., the Sieve of Eratosthenes)
    \item **Meissel-Lehmer algorithm**, which divides the computation into smaller, manageable parts
    \item **Lagarias-Miller-Odlyzko method**, which uses integral approximations
\end{itemize}
Modern implementations allow computation of $\pi(x)$ for very large $x$, with values exceeding $10^{18}$ computed in practice.

\section{References}
For further reading and more detailed proofs, consult:
\begin{itemize}
    \item Dusart, P. (2010). Estimates of some functions over primes without Riemann Hypothesis.
    \item Wikipedia page on the Prime Counting Function: \url{https://en.wikipedia.org/wiki/Prime-counting_function}
    \item TME-EMT Article on Prime Counting Function: \url{https://tmeemt.github.io/Chest/Articles/Art01.html}
\end{itemize}
