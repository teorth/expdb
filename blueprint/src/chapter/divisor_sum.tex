\chapter{The generalized Dirichlet divisor problem}

\begin{definition}[Divisor sum exponents]\label{divisor-def} Let $k \geq 1$. Then $\alpha_k$ is the best exponent for which one has the asymptotic
$$ \sum_{n \leq x} d_k(n) = x P_{k-1}(\log x) + O(x^{\alpha_k+o(1)})$$
as $x \to \infty$, where $P$ is an explicit polynomial of degree $k-1$ and $d_k(n) \coloneqq \sum_{n_1 \dots n_k=n} 1$ is the $k^{\mathrm{th}}$ divisor function.
\end{definition}

\begin{lemma}[$d_1$ exponent] One has $\alpha_1=0$.
\end{lemma}

\begin{lemma}\cite[Theorem 13.1]{ivic} One has $\alpha_2 \leq 35/108$.
\end{lemma}

\begin{lemma}\cite{kolesnik} One has $\alpha_3 \leq 43/96$.
\end{lemma}


\begin{lemma}[Lower bound] $\alpha_k \geq \frac{1}{2} - \frac{1}{2k}$ for all $k$.
\end{lemma}

\begin{proof} See \cite{hardy_divisor_1916, szego-walfisz-I, szego-walfisz-II}.
\end{proof}

It is conjectured that this lower bound is in fact an equality.

\begin{lemma}\label{mas}  Let $k \geq 2$ be an integer. If $M(\sigma,k) = 1$ then $\alpha_k \leq \sigma$.
\end{lemma}

\begin{proof}  See \cite[\S 13.3]{ivic}.
\end{proof}

Using Lemma \ref{mas}, the following bounds were obtained:

\begin{theorem}\cite[Theorem 13.12]{ivic}  One can bound $\alpha_k$ by
    \begin{align*}
        (3k-4)/4k & \hbox{ for } 4 \leq k \leq 8 \\
        35/54 & \hbox{ for } k = 9 \\
        41/60 & \hbox{ for } k = 10 \\
        7/10 & \hbox{ for } k = 11 \\
        (k-2)/(k+2) & \hbox{ for } 12 \leq k \leq 25 \\
        (k-1)/(k+4) & \hbox{ for } 26 \leq k \leq 50 \\
        (31k-98)/32k & \hbox{ for } 51 \leq k \leq 57 \\
        (7k-34)/7k & \hbox{ for } k \geq 58.
    \end{align*}
\end{theorem}


{\bf list known bounds on $\alpha_k$}